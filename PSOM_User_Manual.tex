\documentclass[12pt,letterpaper,titlepage]{article}


% LIST OF PACKAGES
\usepackage{setspace}
\usepackage{graphicx}
\usepackage{times}              % Improve font over the default
\usepackage{amssymb,amsmath}		% Package for mathematical equations
\usepackage{color}
\usepackage{caption,booktabs,array}
\usepackage{microtype}          % Subtly improves word spacing
\usepackage{hyperref}
\usepackage{listings}			% Displays code lines (highly customizable)
\definecolor{codegreen}{rgb}{0,0.6,0}
\definecolor{codegray}{rgb}{0.4,0.4,0.4}
\definecolor{codepurple}{rgb}{0.58,0,0.82}
\definecolor{backcolour}{rgb}{0.95,0.95,0.92}
\lstdefinestyle{mystyle}{
	backgroundcolor=\color{backcolour},   
	commentstyle=\color{codegray},
	keywordstyle=\color{magenta},
	numberstyle=\tiny\color{codegray},
	stringstyle=\color{codepurple},
	basicstyle=\small,
	basewidth=0.5em,
	breakatwhitespace=false,
	breaklines=true,
	captionpos=b,
	keepspaces=true,
	numbers=left,
	numbersep=5pt,
	showspaces=false,
	showstringspaces=false,
	showtabs=false,                  
	tabsize=1,
	aboveskip = 10pt,
	belowskip = 15pt
}
\lstset{style=mystyle}

% MACROS
% The FIXME macro
\newcommand{\fixme}[1]{\color{red}$<$\textbf{FIX ME: #1}$>$\color{black}}

% GLOSSARY
\usepackage[nonumberlist, nopostdot]{glossaries}
% Generate the glossary
\makeglossaries

% MAIN DOCUMENT
\begin{document}
	
	%Term definitions
\newglossaryentry{NI}{name=NI, description={$\sim$}}
\newglossaryentry{NJ}{name=NJ, description={$\sim$}}
\newglossaryentry{NK}{name=NK, description={$\sim$}}
\newglossaryentry{wz}{name=wz, description={$\sim$}}
\newglossaryentry{dtf}{name=dtf, description={non-dimensional time step}}
\newglossaryentry{wsink}{name=wsink, description={Particle sinking velocity}}
\newglossaryentry{stress_top_x}{name=stress\_top\_x, description={Surface wind stress in x-direction}}
\newglossaryentry{stress_top_y}{name=stress\_top\_y, description={Surface wind stress in y-direction}}
\newglossaryentry{stress_top}{name=stress\_top, description={Surface wind stress magnitude}}
\newglossaryentry{wtotal}{name=wtotal, description={Total particle vertical velocity (advection + sinking)}}




	
	\title{User Manual for PSOM}
	\author{$\sim$ et. al}
	\date{2017}
	\maketitle

\tableofcontents
\pagebreak

\section{Introduction}


FROM README FILE :

* GENERAL DESCRIPTION

PSOM, pronounced "soam" (the nectar derived from the churning of the oceans in Indian mythology), stands for Process Study Ocean Model. 
It is a versatile, three-dimensional, non-hydrostatic, computational fluid dynamical model for oceanographic (as well as other) applications (Mahadevan et al., 1996a,b). 
The model uses the finite volume method on a structured grid with boundary fitted coordinates (topography conforming sigma grid in the vertical, and boundary conforming in the horizontal). 
The model has a free-surface. It can be used for large- and small-scale phenomena and can be run in hydrostatic or non-hydrostatic mode (Mahadevan, 2006). It uses a highly efficient solution procedure that exploits the skewness arising from the small geometrical aspect (depth to length scale) ratio of the ocean to speed up the solution of the non-hydrostatic pressure, which is solved by the multigrid method. 

The model has been used for a number of process studies, including investigation of the vertical transport of nutrients for phytoplankton production (Mahadevan and Archer, 2000) and the dynamics of submesoscale processes (Mahadevan and Tandon, 2006; Mahadevan, Tandon and Ferrari, 2010). Since the non-hydrostatic model is well-posed with open boundaries, it can be used as a nested high-resolution model with time-varying boundary conditions applied from a coarser resolution general circulation model (Mahadevan and Archer, 1998). The model is thus ideally suited for high-resolution, limited-region modeling studies. 




\section{Headstart}

FROM GET\_STARTED\_1

\begin{lstlisting}[language=TeX, breaklines]
Here is the minimal set of actions required prior to starting using psom.

* step 1: Select the compiler you want to use. To do that, please edit the ./optfile to set the compiler you will use:       fcomp=...  (ifort, pgf95 etc.). If compiler is modified, the user is encouraged to scan through the ./optfile to make other necessary modifications (e.g., the directory of the gotm library if activated, the define\_parallel flag, etc.).

* step 2: By default, psom will produce both binary files and netcdf files. If you do not have netcdf library installed or smply do not want to use netcdf files, 

- edit the file ./optfile          and set define\_netcdf to F.
- edit the file ./model/inc/cppdef and undefine output\_netcdf variable: 
\#define file\_output\_cdf   by default
\#undef file\_output\_cdf    if you do not want to use netcdf.

These two actions will disable the netcdf output. The executable will only produce binary files.
\end{lstlisting}

\section{Configure, compile, run sequence}

FROM GET\_STARTED\_1

\begin{lstlisting}[language=TeX, breaklines]

Once you have done that, you simply follow the "configure, compile, run" sequence:

* step 3: Configure.

sh tools/configure.sh

* step 4: Compile.

sh tools/compile.sh

* step 5: Run.

./exe/nh $<$ ./namelist\_default


+  ADDITIONAL INFORMATION

Some explanations:

* About step 3.
- Step 3 will search for makedepend (which is necessary to create the makefile). The script will stop if makedepend is not installed. The makefile file will only be created if makedepend can be called.
-> Solution: download makedepend, which is free and widely accessible. For instance, on mac, you can simply sudo port install makedepend (if you have macport installed). 
- Furthermore, if you chose to use netcdf, an attempt to use nc-config will be tried. Note that the code has only been tested with netcdf4 libraries.
-> if nc-config is present, it will be used in order to set the links to netcdf libraries appropriately.
-> if not, you can complete this step yourself by linking the executable to the netcdf libraries. To do so, edit ./optfile before going through step 3.

Step 3 has two effects:
- ./optfile is modified ,
- ./namelist is customized.

To sum up, if step 3 fails, the easiest way is to install makedepend and/or to install netcdf libraries (in case you want to use netcdf output). After having done that, repeat step 3. 


* Step 4 will 
- create the makefile in ./mkfile
- compile

A failure during step 4 most likely indicates that there is an error in the fortran code.


* Step 5 will start and run the model.
By default, the namelist file contains the values of key parameters of the model, that you are likely willing to change. You can of course edit namelist to your convenience and repeat step 5.
By default, messages will appear on the screen and the output files will go to ../output/.
\end{lstlisting}

\section{Particle tracking in PSOM}
PSOM offers the option to release and track particles "online" (i.e., as pat of the numerical simulation). To activate this option, the allow\_particle options must be defined in \texttt{inc/cppdefs.h} by including (See Section \ref{sec: cppdefs}):
\begin{lstlisting}[	basicstyle=\footnotesize, numbers=none]
#define allow_particle
\end{lstlisting}
\subsection{Non-sinking particles}
While key particle parameters are set in \texttt{namelist} (e.g., particle number, frequency of outputs, etc.; see Section \ref{sec: myparticletracking}), the seeding and advection of particles is controlled by the code included in \texttt{particles.f90}. The file includes 6 subroutines:

\begin{itemize}
	\item[\textit{open\_ parti\_ files}:] This subroutine creates the output files where the particle characteristics will be saved. The number of output files is specified in \texttt{namelist} and must be a factor of the total number of particles ($NPR$). Increasing the number of output files reduced the size of the individual files, which proves to be useful when dealing with a very large number of particles. By default, the output files are unformatted binary files.
	\item[\textit{save\_ parti}:] This subroutine loops through all the particles and writes the specified variables. The number of variables saved is important, as it must be known to read the unformatted binary output files.
	\item[\textit{ini\_ particles}:] This subroutine is called when the model timestep matches the particle initialization timestep specified in \texttt{namelist}. This is where the seeding of particle is defined. By default, all  particles are  released at the surface in the middle of the model domain \fixme{random release of particles?}.
	\item[\textit{get\_ parti\_ vel}:] This subroutine interpolates the physical model's velocity field onto the particles' positions (using \textit{interp\_trilinear}; see below). The \textit{get\_parti\_vel} subroutine also interpolates variables of interest onto the particles position (e.g., salinity, temperature, density, vorticity, etc.).
	\item[\textit{parti\_ forward}:] This subroutine extrapolates the position of a particle at the next timestep t+1 using a 2$^{nd}$ order Adams-Bashforth scheme. As an example, the position of the particle in the zonal direction is computed using:
	
	\begin{equation*}
(i,j,k)_{t+1}= (i,j,k)_{t} + dtf \times \frac{1}{2}[3(u,v,w)_{t+1} - (u,v,w)_{t}]
\end{equation*}
	Where $(i,j,k)$ is the position of the particle in the model space, $dtf$ is the non dimensional model time step, $(u,v,w)$ is the three-dimensional velocity of the particle and the subscript represent the corresponding timestep. The corresponding code appears as (e.g., for the particle position in the zonal direction):
	\begin{lstlisting}[language=fortran]
! Assign i-position to particle.
parti(i)%i = parti(i)%i + 0.5d0 * dtf * (3d0 * parti(i)%u - parti(i)%u0)
\end{lstlisting}
	At t=0, the velocities are assumed to be zero (set in \textit{ini\_particles}, and the 2$^{nd}$ order Adams-Bashforth scheme simplifies to a one-step Euler scheme.
	
	\item[\textit{interp\_linear}]	This subroutines is used to interpolate the physical model variables onto a particle's position.
\end{itemize}

\subsection{Sinking particles}

The subroutine \textit{get\_ parti\_ vel} includes the possibility to prescribe a vertical sinking velocity to the particles (set to 0 m/s by default). The prescribed velocity must be scaled appropriately to match the scaling used by PSOM (see Section \fixme{theory? Mahadevan 1996?}). Although not implemented in the code, a horizontal velocity (e.g., to simulate swimming behavior) could also be prescribed to the particles following a similar method.

\subsection{Reading unformatted binary output files}
Particle-tracking output files are written as unformatted binary files. Information on how the output file is built is therefore required to be able to access the data. Two MATLAB routines to import or convert the particle-tracking output are provided with the model code:
\begin{enumerate}
	\item [--] \texttt{particle\_open\_bin.m}: Imports the particle-tracking data into MATLAB as a 3D matrix (Nbr of particles, model time, \# of recorded variables). WARNING: This routine SHOULD NOT BE USED FOR LARGE FILES otherwise the 3D matrix will become too large and will crash MATLAB.
	\item [--] \texttt{particle\_open\_bin.m}: Converts the particle-tracking data into a CSV-file. This can be helpful when trying to import the particle-tracking data into another software (i.e., into an SQL database).
	\end{enumerate}

\section{Test Cases}
\subsection{Wiggles}
\label{sec: Wiggles}
\begin{lstlisting}
- wiggle      : This is a testcase with a wiggling front over a flat topography.  
\end{lstlisting}

\subsection{NA}
\label{sec: NA}
\begin{lstlisting}
- NA          : This is a much more complex simulation of three fronts that go unstable. This casse has particles and tracers for biology.
\end{lstlisting}

\subsection{Shelfbreak}
\label{sec: Shelfbreak}
\begin{lstlisting}
- shelfbreak  : Simulation of the Middle Atlantic Bight shelfbreak with a shelf front and a shelfbreak font. The topography includes a sharp slope at the break. It shows how to use the "user" namelist. 
\end{lstlisting}

\section{Setting up your PSOM simulation}

Write a note about the superceding trick

\subsection{Create your experiment directory}
For every experiment you want to conduct, create a directory that will contain the source files specific to this experiment. As a example, you want to create an experiment named "my\_experiment". First, create a directory \texttt{code/my\_experiment}, in which you will create two subdirectories \texttt{/inc} and \texttt{/src} . You can either create this directory manually, or, in \texttt{code/}, you can run:

\begin{lstlisting}[language=sh]
# Copies the template directory
cp -r expe_template my_experiment
\end{lstlisting}
Whether you are a user or a developer, \textbf{you are strongly invited to leave untouched the files contained in \texttt{code/model/}}. This directory is designed to contain the latest version of the model, which is common to every user at a given time. For every routine that will be specific to \texttt{my\_experiment}, a new subroutine should be created in \texttt{code/my\_experiment/src}. This can be achieved using the following command:
\begin{lstlisting}[language=sh]
# Copies the initial conditions subroutine
cp model/src/ini_st.f90 my_experiment/src
\end{lstlisting}
The compiling step (\fixme{see Section on compiling}) includes a superseding procedure that will take into account the new version of \texttt{ini\_st.f90} (in \texttt{code/my\_experiment/src}) and disregard the standard version found in \texttt{code/model/src}. More precisely, it will create the makefile based on this new state of the model (in \texttt{./mkfile}), compile and create the executable \texttt{code/exe/nh\_my\_experiment}. More details on \texttt{compile.sh} may be found by running:

\begin{lstlisting}[language=sh]
# Provides more information on compile.sh
sh tools/compile.sh -help
\end{lstlisting}

\subsection{Defining your model grid}

The grid size is defined in \texttt{size.h}. If you wish to modify the grid size, you must first copy \texttt{size.h} into your experiment's directory:

\begin{lstlisting}[language=sh]
# Copies the grid file
cp model/inc/size.h my_experiment/inc
\end{lstlisting}

Grids used in previous experiments are listed in this file and commented out. If your grid appears in a commented line, Comment the uncommented line and uncomment the one you want. Be aware that if your grid set requires more than 2Go, you might experience compilation issues. If so, you may fix the issue by editing \texttt{tools/genmakefilel} to replace the default compiling options by:

\begin{lstlisting}[	basicstyle=\footnotesize, numbers=none]
 fflags_o="-fpp -real-size 64 -mcmodel medium -shared-intel -stand 03 -u" 
 fflags_e="-fpp -real-size 64 -mcmodel medium -shared-intel -stand 03 -u" 
\end{lstlisting}

If your grid set does not appear in \texttt{my\_experiment/inc/size.h}, you can create the required line. Defining the model grid is not straight-forward, because of the multi-grid solver mgrid \fixme{(See Section)}. The multi-grid solver is used to allow the reuse of array space in \texttt{mgrid.f90} \fixme{insert link to function?}. Although this issue could now be circumvented by making use of f90's dynamic allocation of memory, the code was originally in fortran77, explaining the need for space re-allocation. A step-by-step approach to defining your own grid is provided below:

\begin{enumerate}
	\item Choose grid dimensions $NI$, $NJ$, and $NK$ (i.e., the number of grid cells in $x$, $y$, and $z$ directions) such that the grid can be subdivided a maximum number of times by a factor of 2 to form "$ngrid$" levels of grid. For example, choosing $NI=48$, $NJ=24$, and $NK=32$ constrains the grid levels to 4 (i.e., $ngrid = 4$), because:

	\begin{align*} 
		NI:& \; 48; 24; 12; 6; 3 &\text{(5 grid levels)}\\
		NJ:& \; 24; 12; 6; 3 &\text{(4 grid levels)}\\
		NK:& \; 32; 16; 8; 4; 2 &\text{(5 grid levels)}\\
	\end{align*}
	The number of grid points possible for a specific $ngrid$ can be computed by multiplying prime numbers (2, 3, 5, 7, etc.) by $2^{ngrid-1}$. Table \ref{tab: ngrid} lists some of the most commonly used number of grid points, depending on the number of grid levels $ngrid$.

	\item Compile \texttt{tools/preproc.f90}:
\begin{lstlisting}[language=sh]
# Compiles preproc.f90 (e.g., using ifort)
ifort preproc.f90 -o preproc
\end{lstlisting}	
	
	\item Runs \texttt{preproc.f90} and fill the values that are asked:
\begin{lstlisting}[language=sh]
# Runs preproc.f90
./preproc.f90
\end{lstlisting}	

\item Copy/Paste the last line the program provides in \texttt{my\_experiment/inc/size.h}. Below is an example for $NI=96$, $NJ=160$, and $NK=32$ (hence $ngrid=5$, see Table \ref{tab: ngrid}):
\begin{lstlisting}[	basewidth=0.5em]
./preproc
	number of grid levels in mgrid, ngrid = 
5
	input the grid info
	NI = 
96
	NJ = 
160
	NK = 
32
	Number of grid points on fine grid: nx,ny,nz   96   160   32
m,ntint,ntout,nbc(m) 1      491520      539784       47104
m,ntint,ntout,nbc(m) 2       61440       73800       11776
m,ntint,ntout,nbc(m) 3        7680       10920        2944
m,ntint,ntout,nbc(m) 4         960        1848         736
m,ntint,ntout,nbc(m) 5         120         384         184
Copy the following line to size.h
INTEGER, PARAMETER :: NI=96, NJ=160, NK = 32, ngrid=5, maxout=626736, maxint=561720, int1=491520
\end{lstlisting}	
\end{enumerate}

\begin{table}[t]
	\caption{\label{tab: ngrid} Number of grid points associated with a specific number of grid levels $ngrid$. These numbers can be computed by multiplying prime numbers (2, 3, 5, 7, etc.) by $2^{ngrid-1}$. Each experiment's number of grid levels is set by the minimum $ngrid$ associated with $NI$, $NJ$, and $NK$.}
	\centering
	\begin{tabular}{ccrrrrrrrr}
		\toprule
		$ngrid$& & \multicolumn{8}{c}{Number of grid points ($NI$, $NJ$, or $NK$)}\\
		\midrule
		4 & &  16  &  24  &  40  &  56  &  88  &  104  &  136  &  152 \\
		5 & &  32  &  48  &  80  &  112  &  176  &  208  &  272  &  304 \\
		6 & &  64  &  96  &  160  &  224  &  352  &  416  &  544  &  608 \\				
		7 & &  128  &  192  &  320  &  448  &  704  &  832  &  1088  &  1216 \\
		\bottomrule
	\end{tabular}
	\label{tab: multiple_regression}
\end{table}

\subsection{cppdefs.h}
\label{sec: cppdefs}
This file defines the different options to be used in the experiment. Again, it is recommended to copy this file into the experiment folder (e.g., \texttt{my\_experiment/inc/} before making any modifications. To include (exclude) an option, use \#define (\#undef) $option\_name$. The file includes 13 options:

\begin{enumerate}
	\item[$runtracmass$ :] placeholder
	\item[$periodic\_ew$ :] placeholder
	\item[$periodic\_ns$ :] placeholder
	\item[$allow\_particle$ :] If defined, allows the seeding of particles in the experiment. Please refer to section \fixme{ref to particle section} for a detailed explanation on particle seeding.
	\item[$rhoonly$ :] If defined, only the density field $rho$ is used. The density field is stored in the salinity array ($s$; see \texttt{evalrho\_rho.f90}). If not defined, $rho$ is computed from the salinity ($s$) and temperature ($T$) fields (see \texttt{evalrho\_sT.f90}).
	\item[$relaxation$ :] placeholder
	\item[$fixed\_bottom\_thickness$ :] placeholder
	\item[$file\_output$ :] placeholder
	\item[$file\_output\_cdf$ :] placeholder
	\item[$file\_output\_bin$ :] placeholder
	\item[$gotm\_call$ :] placeholder
	\item[$implicit$ :] placeholder
	\item[$parallel$ :] placeholder
\end{enumerate}


\subsection{namelist}
This file defines key parameters relating to the experiment (e.g., grid resolution, time step, diffusion, output, ...). Again, it is recommended to copy this file into the experiment folder (e.g., \texttt{my\_experiment/} before making any modifications. Each parameter in the file is either self-explanatory or include a short description as a comment.

\subsection{Defining the initial conditions}
Initial conditions can be specified either in the corresponding subroutines, or from an input file. The former approach is used in the Shelfbreak test-case (see Section \ref{sec: Shelfbreak}), where the temperature and salinity distributions are determined from analytical expressions in DO-loops, and only requires a limited knowledge of the model grid. The latter approach can sometimes be more practical, especially when using available data products to initialize the model. However, this approach requires mapping the data used to initialize the experiment to the pre-defined model grid. 

The horizontal grid is relatively straightforward to determine, given the grid size specified in \texttt{my\_experiment/inc/size.h} (i.e., $NI$ and $NJ$), and the grid resolution specified in \texttt{my\_experiment/namelist} (i.e., $dx$ and $dy$). The location of each grid point can be computed using the following equations:
\begin{align}
x(i) &= -dx/2 + i dx; &i &= (0, 1 , 2 \ldots NI, NI+1)\\
y(j) &= -dy/2 + j dy; &j &= (0, 1 , 2 \ldots NJ, NJ+1)
\end{align}

\subsubsection{Temperature, salinity, and density}
The initial conditions in the temperature and salinity

\subsection{Wind stress}
To specify a customized wind forcing, the code in \texttt{wind\_stress.f90} can be modified. The wind stress at the surface is prescribed to the model through the variables $stress\_top\_x$, $stress\_top\_y$, and $stress\_top$. The dimensions of these three variables are ($NI$,$NJ$) (see \texttt{header.f90}). The surface wind stress can be read from a file: 

\begin{lstlisting}[	basewidth=0.5em, language=fortran]
! Import the wind stress time series for model forcing
if (step.eq.1) then
		open(unit=17, file='youfilefullpath.in')
		do i=1,nsteps
				read(17,fmt="(F5.10,F5.10)") stressxTS(i),stressyTS(i)
		end do
		close (17)
		PRINT*,"Read wind stress"
end if
stress_top_x = stressxTS(step)
stress_top_y = stressyTS(step)
\end{lstlisting}	
or specified as a constant:

\begin{lstlisting}[	basewidth=0.5em, language=fortran]
stress_top_x = 0.05d0
stress_top_y = 0.01d0
PRINT*,"Read Wind Stress"
\end{lstlisting}	

If your domain includes solid boundary (i.e., no periodicity), it is recommended to damp the surface wind stress close to the boundaries, to avoid upwelling/downwelling. Below is an example of wind stress damping at the north/south boundaries using a tanh profile. A similar approach can be used in the east/west direction.

\begin{lstlisting}[	basewidth=0.5em, language=fortran]
! Apply a tanh profile in the meridional direction
ycenter = 0.5*(yc(NJ)+yc(1))	! Find the middle of the domain
ywindmin = 10.0 							! Starts damping 10 km from southern boundary
ywindmax = yc(NJ)-10.d0   	   ! Starts damping 10 km from northern boundary
edge = 0.06    							! tightnness of the padding in the wind stress

do j=1,NJ
		if (yc(j).lt.yc(NJ/2)) then
				stressprofile(j) = 0.5*(tanh(edge*(yc(j)-ywindmin)*PI)+1.d0)
		else
				stressprofile(j) = -0.5*(tanh(edge*(yc(j)-ywindmax)*PI)-1.d0)
		end if
end do

do j=1,NJ
		do i=1,NI
				stress_top_x(i,j) = stressxTS(step)*stressprofile(j)
				stress_top_y(i,j) = stressyTS(step)*stressprofile(j)
		end do
end do
\end{lstlisting}	

\subsection{Surface Heat Fluxes}



\subsection{Particle-tracking in PSOM}
\label{sec: myparticletracking}
To set up a particle-tracking experiment in PSOM, the allow\_particle option must be defined in \texttt{inc/cppdefs.h} (See Section \ref{sec: cppdefs}):
\begin{lstlisting}[	basicstyle=\footnotesize, numbers=none]
#define allow_particle
\end{lstlisting}
Four key variables related to particle-tracking are set in the \texttt{namelist} file:
\begin{enumerate}
	\item The total number of Particles ($NPR$). $NPR$ must be a multiple of the number of output files (see below).
	\item The time step at which the particles are released ($ini$\_$particle$\_$time$). $ini$\_$particle$\_$time$ \textbf{must be} greater than 0, or than $pickup$\_$step$ if pickup files are used to initialize the experiment (see Section \ref{sec: namelist} \fixme{Refer to section about namelist and pick up files}). If $ini$\_$particle$\_$time$ = $pickup$\_$step$, no particle output file will be written.
	\item The number of output files to generate ($parti$\_$file$\_$num$). Increasing the number of files logically decreases the file size. This is especially useful when dealing with a very large number of particles, or when writing particles' position at high frequency. The number of file must be a factor of $NPR$ (see above).
	\item The frequency of particle outputs ($parti$\_$outfreq$) in number of time steps.
\end{enumerate}

The seeding and tracking of the particles in PSOM are controlled by subroutines located in \texttt{particles.f90}. To personalize the seeding of particles in the model, the code in the subroutine $ini$\_$particles$ should be altered. The subroutine $save$\_$parti()$ can be edited to customize the number of variables written to the particle output file. To remove a variable from the output files, simply comment out the corresponding line. %Variables can also be added to the list, although it will require a better understanding of how the physical and particle models operate, as many other changes to the code


%Print the glossary
\pagebreak
\glsaddall				% Include all glossary entries
\glossarystyle{listhypergroup}	% Define style for glossary
\renewcommand*{\arraystretch}{0.3}% default is 1
\printglossaries

\end{document}

