\begin{lstlisting}[language=matlab]
clear

% ** This code imports the particle-tracking data directly from bin output
% files. 
% ** The file path has to be specified.
% ** output is a 3D matrix where the 1st dimension is the particle number,
% the 2nd dimension is the model time, and the 3rd diemnsion is the
% recorded variables.

%%%%%%%%%%%%%%%%%%%%%%%%
% WARNING
%%%%%%%%%%%%%%%%%%%%%%%%
% This routine SHOULD NOT BE USED FOR LARGE FILES otherwise MATLAB matrices
% will become too large and will crash MATLAB.
%%%%%%%%%%%%%%%%%%%%%%%%

%%
%%%%%%%%%%%%%%%%%%%%%%%%
% PARAMETERS
%%%%%%%%%%%%%%%%%%%%%%%%

% number of files to import into matlab
number_of_files = 1;
% filepath
path = 'specify the path for the output files here';
% Number of variables recorded (see particles.f90)
number_of_variables = 20;

%%%%%%%%%%%%%%%%%%%%%%%%
% CORE CODE
%%%%%%%%%%%%%%%%%%%%%%%%

% Loops through the nummber of files to open
for filenum = 1:number_of_files

% Create fullpath
filename = ['op.parti-',num2str(filenum,'%03.f'),'.bin'];
fullpath = [path,'/',filename];

% Display the file being extracted
disp(filename);

% Open the file
fileID = fopen(fullpath);

% Extract the data (refer to particles.f90 to confirm that number) 
A = fread(fileID,[number_of_variables Inf],'double');
A = A';

% Finds the number of particle per file using the ID numbers
if filenum == 1
partnum = max(A(:,1));
end

% write a matrix of dimensions:
% # of particles x timestep x recorded variables
for Np = (filenum-1)*partnum+1:(filenum-1)*partnum+partnum
ind = find(A(:,1) == Np);
data(Np,:,:) = A(ind,:);
end; clear Np ind

fclose(fileID);
clear A fileID filename path fullpath ans

end; clear filenum partnum
\end{lstlisting}